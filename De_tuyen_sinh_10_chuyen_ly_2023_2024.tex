\documentclass[15pt]{article}
\usepackage[utf8]{vietnam}
\usepackage{amsthm}
\usepackage{amsmath}
\usepackage{graphicx} % Required for inserting images

\title{Lời giải kham khảo đề thi chuyên lý tuyển sinh lớp 10\\ 2023 - 2024}
\begin{document}

\maketitle

\section*{Lời giải đề xuất bởi các thành viên:}
- \textit{Châu Nguyễn Thanh Duy}, LK14, trường THPT chuyên Nguyễn Quang Diêu\\
- \textit{Quách Khả Đăng}, LK14, trường THPT chuyên Nguyễn Quang Diêu\\
- \textit{Võ Minh Hào}, LK14, trường THPT chuyên Nguyễn Quang Diêu

\section*{Câu 1: (2,0 điểm)}
a) Tính AC và AB.\\
\\
Quãng đường người đi xe đạp đi được trong 1h là:\\
\begin{equation}
    s_{2} = v_{2}.t_{2} = 15.1 = 15 (km)
\end{equation}
Mà từ đề bài ta có:\\
\begin{equation}
    s_{2} = \frac{3}{4}AC
\end{equation}
Từ (1) và (2) ta có:
\begin{equation*}
    s_{2} = \frac{3}{4}AC = 15 \\
    \implies AC = 20 (km)
\end{equation*}
Chọn mốc thời gian lúc người đi bộ xuất phát.\\
Thời gian người đi bộ đến B là:
\begin{equation*}
    t_{1} = \frac{BC}{5} + \frac{1}{2}
\end{equation*}
Thời gian người đi xe đạp đến B là:
\begin{equation*}
    t_{2} = \frac{AC}{15} + \frac{BC}{15} + 1 = \frac{20}{15} + \frac{BC}{15} + 1
\end{equation*}
Mà 2 người đều đến B cùng lúc nên ta có: \\
\begin{equation*}
    \begin{aligned}
        & t_{1} = t_{2} \\
        & \implies BC = 13,75 (km) \\
        & \implies AB = AC + BC = 20 + 13,75 = 33,75 (km)
    \end{aligned}
\end{equation*}
\\
\\
b) Tìm vận tốc của người đi xe đạp.\\
\\
Gọi D là điểm mà người đi bộ bắt đầu nghỉ.\\
\begin{equation*}
    AD = AC + v_{1}t = 20 + 5.2 = 30 (km)
\end{equation*}
Vận tốc để  2 người gặp nhau khi người đi bộ chuẩn bị xuất phát sau khi nghỉ 30', tức là sau 1,5h kể từ khi người đi xe đạp xuất phát là:\\
\begin{equation*}
    v = \frac{AD}{t_{2}} = \frac{30}{1,5} = 20 (km/h)
\end{equation*}
Vậy để gặp người đi bộ lúc người ấy đã nghỉ xong thì người đi xe đạp phải có vận tốc: $v = 20 (km/h)$

\section*{Câu 2: (1,0 điểm)}
Ta có tổng khối lượng của nước là:\\
\begin{equation*}
    m = m_{1} + m_{2} (1)
\end{equation*}
Ta có phương trình cân bằng nhiệt:
\begin{equation*}
    \begin{aligned}
        & Q_{thu} = Q_{tỏa}\\
        & \leftrightarrow m_{2}c(t_{2} - t) = m_{1}c(t - t_{1}) \\
        & \leftrightarrow m_{2}(t_{2} - t) = m_{1}(t - t_{1}) \\
        & \leftrightarrow m_{2}c(84 - 36) = m_{1}c(36 - 12) \\
        & \leftrightarrow 48m_{2} = 24m_{1} (2)
    \end{aligned}
\end{equation*}
Từ (1) và (2) ta có:
\begin{cases}
    m_{1} = 2 (kg) \\
    m_{2} = 1 (kg) \\
\end{cases}\\
\\
Vậy để có được 3kg nước ở nhiệt độ $36^o C$ thì ta cần 2kg nước ở nhiệt độ $12^o C$ và 1 kg nước ở nhiệt độ $84^o C$.

\section*{Câu 3: (2,5 điểm)}
a) Khi K mở ta có: $[(R_{1} nt R_{3}) // (R_{2} nt R_{4}] nt R_{5}$\\
\begin{equation*}
    \begin{aligned}
        & R_{13} = R_{1} + R_{3} = 1 + 3 = 4 (\Omega)\\
        & R_{24} = R_{2} + R_{4} = 4 + 8 = 12 (\Omega)\\
        & R_{1234} = \frac{R_{13}R_{24}}{R_{13} + R_{24}} = \frac{4.12}{4 + 12} = 3 (\Omega) \\
        & R_{k mở} = R_{1234} + R_{5} = 3 + 3 = 6 (\Omega) \\
    \end{aligned}
\end{equation*}
Ta có:
\begin{equation*}
    U_{CD} = U_{V} = U_{CA} + U_{AD} = -U_{1} + U_{2}
    \leftrightarrow U_{2} - U_{1} = 0,5 (V) (1)
\end{equation*}
Ta lại có:
\begin{equation*}
    U_{13} = U_{24} = U - IR_{5} (2)
\end{equation*}
Mà:
\begin{equation*}
    I = \frac{U}{R_{k mở}} = \frac{U}{6} (A)
\end{equation*}
Thay vào (2):
\begin{equation*}
    \implies U_{13} = U_{24} = U - \frac{U}{6}.R_{5} = \frac{1}{2}U (V)
\end{equation*}
Ta có:
\begin{equation*}
    \begin{aligned}
        U_{1} & = U_{13} - I_{3}R_{3} = U_{13} - \frac{U_{13}}{R_{13}}.R_{3}
              & = \frac{U}{2} - \frac{U}{2.4}.3 = \frac{1}{8}U (V)
    \end{aligned}
\end{equation*}
Ta tiếp tục có:
\begin{equation*}
    \begin{aligned}
        U_{2} & = U_{24} - I_{4}R_{4} = U_{24} - \frac{U_{24}}{R_{24}}.R_{4}
              & = \frac{U}{2} - \frac{U}{2.12}.8 = \frac{1}{6}U (V)
    \end{aligned}
\end{equation*}
Thay các giá trị $U_{1}$ và $U_{2}$ vào (1) ta có:\\
\begin{equation*}
    \begin{aligned}
        & \frac{1}{6}U - \frac{1}{8}U = 0,5
        \implies U = 12 (V)
    \end{aligned}
\end{equation*}
b) Khi k đóng ta có $[(R_{1} nt R_{3}) // (R_{2} nt R_{4}) // R_{6}] nt R_{5}$\\
\begin{equation*}
    R_{12346} = \frac{R_{1234}.R_{6}}{R_{1234} + R_{6}} = \frac{3R_{6}}{3 + R_{6}} (\Omega)
\end{equation*}
\begin{equation*}
    R_{kd} = R_{12346} + R_{5} = \frac{3R_{6}}{3 + R_{6}} + 3 = \frac{6R_{6} + 9}{3 + R_{6}}
\end{equation*}
Ta có:
\begin{equation*}
    \begin{aligned}
        U_{13} & = U_{24} = U_{6} = U - I'R_{5} = U - \frac{U}{R_{kd}}.R_{5} \\
               & = U - \frac{U(3 + R_{6}).3}{6R_{6} + 9}.3 \\
               & = U(1 - \frac{3 + R_{6}}{2R_{6} + 3} \\
               & = U.\frac{R_{6}}{2R_{6} + 3} \\
    \end{aligned}
\end{equation*}
Ta có:
\begin{equation*}
P_{6} = \frac{U_{6}^2}{R_{6}} = \frac{U^2.R_{6}^2}{(2R_{6} + 3)^2.R_{6}} = \frac{U^2R_{6}}{(2R_{6} + 3)^2}\\
\end{equation*}
\begin{equation*}
    \implies 3,84 = \frac{12^2.R_{6}}{(2R_{6} + 3)^2}\\
\end{equation*}
\begin{equation*}
    \leftrightarrow 15,36R_{6}^2 - 97,92R_{6} + 34,56 = 0\\
\end{equation*}
$\implies R_{6} = \frac{3}{8} (\Omega)$ (Loại vì $R_{6}$ có giá trị nguyên)\\
$\implies R_{6} = 6 (\Omega)$ (Nhận)\\

\section*{Câu 4: (1,5 điểm)}
a) Tính R.\\
\\
Ta có: $P_{hp} = I^2R = \frac{P^2}{U^2}R$\\
\\
$\implies R = \frac{P_{hp}}{P^2}U^2 = \frac{0,026P}{P^2}U^2 = \frac{0,026}{P}U^2$\\
\\
$\implies R = \frac{0,026}{160.10^3}.(6.10^3)^2 = 5,85 (\Omega)$\\
\\
b) Tìm x = $\frac{n1}{n2}$\\
\\
Ta có công suất hao phí lúc này là:
\begin{equation*}
    P_{hp'} = 1040 (W)
\end{equation*}
Ta lại có: $P_{hp'} = I'^2R$\\
Cường độ dòng điện truyền tải qua dây khi đó là:
\begin{equation*}
    I' = \sqrt{\frac{P_{hp'}}{R}} = \sqrt{\frac{1040}{5,85}} = \frac{40}{3} (A)
\end{equation*}
Hiệu điện thế đường dây là:
\begin{equation*}
    \Delta U = I'R = \frac{40}{3}.5,85 = 78 (V)
\end{equation*}
Công suất tiêu thụ là:
\begin{equation*}
    P' = P - P_{hp'}
\end{equation*}
Hiệu điện thế nới tiêu thụ là:
\begin{equation*}
    U' = \frac{P'}{I'} = \frac{P - P_{hp'}}{I'} = \frac{160.10^3 - 1040}{\frac{40}{3}} = 11922 (V)
\end{equation*}
Hiệu điện thế cuộn thứ cấp là: $U_{2}$ = U' + $\Delta U$ = 11922 + 78 = 12000 (V)\\
Hiệu điện thế cuộn sơ cấp là: $U_{1} = U = 6.10^3 (V)$\\
Vậy tỉ số máy biến thế khi đó là:
\begin{equation*}
    x = \frac{n_{1}}{n_{2}} = \frac{U_{1}}{U_{2}} = \frac{6.10^3}{12.10^3} = \frac{1}{2}
\end{equation*}

\section*{Câu 5: (2,0 điểm)}
a)
Xét $\triangle OHS \cong \triangle OH'S'$ ta có:\\
\begin{equation*}
    \implies \frac{OH}{OH'} = \frac{HS}{H'S'}
    \leftrightarrow \frac{d}{d'} = \frac{h}{h'} (1)
\end{equation*}
Xét $\triangle OIF \cong \triangle HSF$ ta có:\\
\begin{equation*}
    \implies \frac{HS}{OI} = \frac{HF}{OF}\\
    \leftrightarrow \frac{HS}{H'S'} = \frac{OH - OF}{OF}
    \leftrightarrow \frac{h}{h'} = \frac{d - f}{f} (2)
\end{equation*}\\
Từ (1) và (2) ta có:\\
\begin{equation*}
    \implies \frac{d}{d'} = \frac{d - f}{f}\\
    \leftrightarrow df = d'd - d'f\\
    \leftrightarrow \frac{1}{f} = \frac{1}{d} + \frac{1}{d'}
\end{equation*}\\
Mà ta có:
\begin{cases}
    f = 20 (cm)\\
    d = 40 (cm)
\end{cases}
$\implies d' = 40 (cm)$\\
Từ (1) $\implies \frac{40}{40} = \frac{3}{h'} \implies h' = 3 (cm)$\\
Vậy khoảng cách từ S' đến thấu kính là 40cm và S' đến trục chính là 3cm.\\
b)
Ta có:
\begin{equation*}
    \frac{1}{f} = \frac{1}{d} + \frac{1}{d'}\\
\end{equation*}
Mà $d_{1} = 30cm$\\
\begin{equation*}
    \implies \frac{1}{20} = \frac{1}{30} + \frac{1}{d'} \implies d' = 60 (cm)
\end{equation*}
Tương tự (1)  ta có:
\begin{equation*}
    \frac{d_{1}}{d_{1}'} = \frac{h}{h''} \leftrightarrow \frac{30}{60} = \frac{3}{h''}\\
    \implies h'' = 6 (cm)
\end{equation*}
Độ dịch chuyển của ảnh S' là:\\
\begin{equation*}
    S'S'' = \sqrt{(d_{1}' - d')^2 + (h'' - h')^2}\\
    \implies S'S'' = \sqrt{409} (cm)\\
\end{equation*}
Vận tốc của ảnh trong quá trình dịch chuyển là:\\
\begin{equation*}
    v' = \frac{S'S''}{t}
\end{equation*}
Mà:
\begin{equation*}
    t = \frac{d_{1} - d}{v} = \frac{40 - 30}{10} = 1 (s)
\end{equation*}
\begin{equation*}
    \implies v' = \frac{\sqrt{409}}{1} = \sqrt{409} (cm/s)
\end{equation*}
Vậy vận tốc của ảnh trong thời gian chuyển động là $\sqrt{409}$ (cm/s).
\section*{Câu 6: (1,0 điểm)}
Phương án xác định khối lượng của thuỷ ngân:\\
Cho lọ thủy ngân lên cân, ta cân được khối lượng tổng cộng:
\begin{equation*}
    m = m_{1} + m_{2} (1)
\end{equation*}
Cho lọ thủy ngân vào bình chia độ chứa đầy nước, ta xác định được thể tích nước dâng lên:
\begin{equation*}
    V = V_{1} + V_{2} = \frac{m_{1}}{D_{1}} + \frac{m_{2}}{D_{2}} (1)
\end{equation*}
Từ (1) và (2), ta có:\\
\begin{equation*}
    m_{2} = \frac{(m - VD_{1})D_{2}}{D_{2} - D_{1}}
\end{equation*}

\end{document}
