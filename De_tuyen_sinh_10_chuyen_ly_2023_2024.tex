\documentclass[50pt]{article}
\usepackage[utf8]{vietnam}
\usepackage[margin=1in]{geometry}
\usepackage{amsthm}
\usepackage{amsmath}
\usepackage{xcolor}
\usepackage{float}
\usepackage{graphicx} % Required for inserting images

\title{\textbf{\color[HTML]{326fbf}LỜI GIẢI THAM KHẢO TUYỂN SINH 10 MÔN VẬT LÝ CHUYÊN TỈNH ĐỒNG THÁP\\ 2023 - 2024}}
\date{Ngày 9 tháng 6 năm 2023}
\begin{document}

\maketitle

\section*{\color[HTML]{4287f5}Lời giải đề xuất bởi các thành viên:}
- \textit{Châu Nguyễn Thanh Duy}, LK14, trường THPT chuyên Nguyễn Quang Diêu\\
\\
- \textit{Quách Khả Đăng}, LK14, trường THPT chuyên Nguyễn Quang Diêu\\
\\
- \textit{Võ Minh Hào}, LK14, trường THPT chuyên Nguyễn Quang Diêu\\
\hline
\section*{\color[HTML]{4287f5}Câu 1: (2,0 điểm)}
Một người đi bộ khởi hành từ cột mốc C đến cột mốc B với vận tốc $v_{1}$ = 5km/h sau khi đi được 2 giờ người ấy ngồi nghỉ 30 phút rồi đi tiếp về B. Người thứ hai đi xe đạp khởi hành từ cột mốc A cũng đi về B như hình vẽ (Hình 1) với vận tốc $v_{2}$ = 15km/h nhưng khởi hành sau người đi bộ 1 giờ (coi các chuyển động là thẳng đều).

a) Tính quãng đường AC và AB. Biết 2 người đó đến B cùng một lúc và khi người đi bộ bắt đầu ngồi nghỉ thì người đi xe đạp đã đi được 3/4 quãng đường AC

b) Để gặp người đi bộ lúc người ấy đã nghỉ xong thì người đi xe đạp phải có vận tốc bằng bao nhiêu?
\begin{figure}[H]
    \centering
    \includegraphics[width=0.5\linewidth]{ABC.jpg}
    \label{fig:enter-label}
\end{figure}
\subsection*{\color[HTML]{4287f5}Lời giải:}
a) Tính AC và AB.\\
\\
Quãng đường người đi xe đạp đi được trong 1h là:\\
\begin{equation}
    s_{2} = v_{2}.t_{2} = 15.1 = 15 (km)
\end{equation}
Mà từ đề bài ta có:\\
\begin{equation}
    s_{2} = \dfrac{3}{4}AC
\end{equation}
Từ (1) và (2) ta có:
\begin{equation*}
    s_{2} = \dfrac{3}{4}AC = 15 \\
    \implies AC = 20 (km)
\end{equation*}
Chọn mốc thời gian lúc người đi bộ xuất phát.\\
Thời gian người đi bộ đến B là:
\begin{equation*}
    t_{1} = \dfrac{BC}{5} + \dfrac{1}{2}
\end{equation*}
Thời gian người đi xe đạp đến B là:
\begin{equation*}
    t_{2} = \dfrac{AC}{15} + \dfrac{BC}{15} + 1 = \dfrac{20}{15} + \dfrac{BC}{15} + 1
\end{equation*}
Mà 2 người đều đến B cùng lúc nên ta có: \\
\begin{equation*}
    \begin{aligned}
        & t_{1} = t_{2} \\
        & \implies BC = 13,75 (km) \\
        & \implies AB = AC + BC = 20 + 13,75 = 33,75 (km)
    \end{aligned}
\end{equation*}
\\
\\
b) Tìm vận tốc của người đi xe đạp.\\
\begin{figure}[H]
    \centering
    \includegraphics[width=0.5\linewidth]{ABCD.jpg}
    \label{fig:enter-label}
\end{figure}
\noindent Gọi D là điểm mà người đi bộ bắt đầu nghỉ.\\
\begin{equation*}
    AD = AC + v_{1}t = 20 + 5.2 = 30 (km)
\end{equation*}
Vận tốc để  2 người gặp nhau khi người đi bộ chuẩn bị xuất phát sau khi nghỉ 30', tức là sau 1,5h kể từ khi người đi xe đạp xuất phát là:\\
\begin{equation*}
    v = \dfrac{AD}{t_{2}} = \dfrac{30}{1,5} = 20 (km/h)
\end{equation*}
Vậy để gặp người đi bộ lúc người ấy đã nghỉ xong thì người đi xe đạp phải có vận tốc: $v = 20 (km/h)$
\section*{\color[HTML]{4287f5}Câu 2: (1,0 điểm)}
Để có 3 kg nước ở nhiệt độ $36^oC$, người ta pha một lượng nước có khối lượng $m_{1}$ (kg) ở nhiệt độ $12^oC$ với một lượng nước có khối lượng $m_{2}$ (kg) ở nhiệt độ $84^oC$. Tính khối lượng $m_{1}$, và $m_{2}$. Bỏ qua sự trao đổi nhiệt với môi trường.
\subsection*{\color[HTML]{4287f5}Lời giải:}
Ta có tổng khối lượng của nước là:\\
\begin{equation*}
    m = m_{1} + m_{2} (1)
\end{equation*}
Ta có phương trình cân bằng nhiệt:
\begin{equation*}
    \begin{aligned}
        & Q_{toa} = Q_{thu}\\
        & \leftrightarrow m_{2}c(t_{2} - t) = m_{1}c(t - t_{1}) \\
        & \leftrightarrow m_{2}(t_{2} - t) = m_{1}(t - t_{1}) \\
        & \leftrightarrow m_{2}(84 - 36) = m_{1}(36 - 12) \\
        & \leftrightarrow 48m_{2} = 24m_{1} (2)
    \end{aligned}
\end{equation*}
Từ (1) và (2) ta có:
\begin{cases}
    m_{1} = 2 (kg) \\
    m_{2} = 1 (kg) \\
\end{cases}\\
\\
Vậy để có được 3kg nước ở nhiệt độ $36^o C$ thì ta cần 2kg nước ở nhiệt độ $12^o C$ và 1 kg nước ở nhiệt độ $84^o C$.
\section*{\color[HTML]{4287f5}Câu 3: (2,5 điểm)}
Cho mạch điện như hình vẽ. Nguồn điện có hiệu điện thế U không đổi, vôn kế có điện trở rất lớn, giá trị các điện trở là $R_{1}$ = 1$\Omega$ $R_{2}$ = 4$\Omega$ $R_{3}$ = $R_{5}$ = 3$\Omega$ $R_{4}$ = 8$\Omega$ Điện trở dây nối và khóa K không đáng kể.

a) Khi khóa K mở, số chỉ của vôn kế $U_{CD}$ = 0,5 V. Tìm điện trở tương đương của toàn mạch và hiệu điện thế U.

b) Khi khóa K đóng, công suất tiêu thụ trên $R_{6}$ là 3,84 W. Tìm điện trở $R_{6}$ Biết $R_{6}$ có giá trị nguyên.
\begin{figure}[H]
    \centering
    \includegraphics[width=0.3\linewidth]{circuit.jpg}
    \label{fig:enter-label}
\end{figure}
\subsection*{\color[HTML]{4287f5}Lời giải:}
a) Khi K mở ta có: $[(R_{1} nt R_{3}) // (R_{2} nt R_{4})] nt R_{5}$\\
\begin{equation*}
    \begin{aligned}
        & R_{13} = R_{1} + R_{3} = 1 + 3 = 4 (\Omega)\\
        & R_{24} = R_{2} + R_{4} = 4 + 8 = 12 (\Omega)\\
        & R_{1234} = \dfrac{R_{13}R_{24}}{R_{13} + R_{24}} = \dfrac{4.12}{4 + 12} = 3 (\Omega) \\
        & R_{k mở} = R_{1234} + R_{5} = 3 + 3 = 6 (\Omega) \\
    \end{aligned}
\end{equation*}
Ta có:
\begin{equation*}
    U_{CD} = U_{V} = U_{CA} + U_{AD} = -U_{1} + U_{2}
    \leftrightarrow U_{2} - U_{1} = 0,5 (V) (1)
\end{equation*}
Ta lại có:
\begin{equation*}
    U_{13} = U_{24} = U - IR_{5} (2)
\end{equation*}
Mà:
\begin{equation*}
    I = \dfrac{U}{R_{k mở}} = \dfrac{U}{6} (A)
\end{equation*}
Thay vào (2):
\begin{equation*}
    \implies U_{13} = U_{24} = U - \dfrac{U}{6}.R_{5} = \dfrac{1}{2}U (V)
\end{equation*}
Ta có:
\begin{equation*}
    \begin{aligned}
        U_{1} & = U_{13} - I_{3}R_{3} = U_{13} - \dfrac{U_{13}}{R_{13}}.R_{3}
              & = \dfrac{U}{2} - \dfrac{U}{2.4}.3 = \dfrac{1}{8}U (V)
    \end{aligned}
\end{equation*}
Ta tiếp tục có:
\begin{equation*}
    \begin{aligned}
        U_{2} & = U_{24} - I_{4}R_{4} = U_{24} - \dfrac{U_{24}}{R_{24}}.R_{4}
              & = \dfrac{U}{2} - \dfrac{U}{2.12}.8 = \dfrac{1}{6}U (V)
    \end{aligned}
\end{equation*}
Thay các giá trị $U_{1}$ và $U_{2}$ vào (1) ta có:\\
\begin{equation*}
    \begin{aligned}
        & \dfrac{1}{6}U - \dfrac{1}{8}U = 0,5
        \implies U = 12 (V)
    \end{aligned}
\end{equation*}
b) Khi k đóng ta có $[(R_{1} nt R_{3}) // (R_{2} nt R_{4}) // R_{6}] nt R_{5}$\\
\begin{equation*}
    R_{12346} = \dfrac{R_{1234}.R_{6}}{R_{1234} + R_{6}} = \dfrac{3R_{6}}{3 + R_{6}} (\Omega)
\end{equation*}
\begin{equation*}
    R_{kd} = R_{12346} + R_{5} = \dfrac{3R_{6}}{3 + R_{6}} + 3 = \dfrac{6R_{6} + 9}{3 + R_{6}}
\end{equation*}
Ta có:
\begin{equation*}
    \begin{aligned}
        U_{13} & = U_{24} = U_{6} = U - I'R_{5} = U - \dfrac{U}{R_{kd}}.R_{5} \\
               & = U - \dfrac{U(3 + R_{6}).3}{6R_{6} + 9}.3 \\
               & = U(1 - \dfrac{3 + R_{6}}{2R_{6} + 3}) \\
               & = U.\dfrac{R_{6}}{2R_{6} + 3} \\
    \end{aligned}
\end{equation*}
Ta lại có:
\begin{equation*}
P_{6} = \dfrac{U_{6}^2}{R_{6}} = \dfrac{U^2.R_{6}^2}{(2R_{6} + 3)^2.R_{6}} = \dfrac{U^2R_{6}}{(2R_{6} + 3)^2}\\
\end{equation*}
\begin{equation*}
    \implies 3,84 = \dfrac{12^2.R_{6}}{(2R_{6} + 3)^2}\\
\end{equation*}
\begin{equation*}
    \begin{aligned}
        & \leftrightarrow 15,36R_{6}^2 - 97,92R_{6} + 34,56 = 0\\
        & \implies \begin{cases}
            R_{6} = \dfrac{3}{8} (\Omega) (loai)\\
            \\
            R_{6} = 6 (\Omega) (nhan)
        \end{cases}
    \end{aligned}
\end{equation*}
\section*{\color[HTML]{4287f5}Câu 4: (1,5 điểm)}
Để truyền tải công suất điện 160 kW từ một nhà máy thủy điện đến nơi tiêu thụ, người ta dùng dây tải điện đồng chất có điện trở R. Biết công suất hao phí trên đường dây tải điện bằng 2,6\% công suất diện truyền đi và hiệu điện thế đầu đường dây tải là 6 kV.

a) Tính điện trở dây tải điện.

b) Trong điều kiện được nêu trên, lắp thêm một máy biến thế ở đầu đường dây truyền tải diện có tỉ số số vòng dây cuộn sơ cấp và thứ cấp là. Tìm x để công suất hao phí trên đường dây tải điện khi đó là 1040W. $\dfrac{n_{1}}{n_{2}}$ = x
\subsection*{\color[HTML]{4287f5}Lời giải:}
a) Tính R.\\
\\
Ta có:
\begin{equation*}
    \begin{aligned}
        & P_{hp} = I^2R = \dfrac{P^2}{U^2}R \\
        & \implies R = \dfrac{P_{hp}}{P^2}U^2 = \dfrac{0,026P}{P^2}U^2 = \dfrac{0,026}{P}U^2 \\
        & \implies R = \dfrac{0,026}{160.10^3}.(6.10^3)^2 = 5,85 (\Omega)
    \end{aligned}
\end{equation*}
b) Tìm x = $\dfrac{n_{1}}{n_{2}}$\\
\\
Ta có công suất hao phí lúc này là:
\begin{equation*}
    P_{hp}' = 1040 (W)
\end{equation*}
Ta lại có:
\begin{equation*}
    \begin{aligned}
        & P_{hp}' = \frac{P^2}{U_{2}^2}.R \leftrightarrow 1040 = \frac{(160.10^3)^2}{U_{2}^2}.5,85\\
        & \implies U_{2} = 12000 (V)
    \end{aligned}
\end{equation*}
Hiệu điện thế cuộn thứ cấp là: $U_{2}$ = 12000 (V)\\
Hiệu điện thế cuộn sơ cấp là: $U_{1} = U = 6.10^3 (V)$\\
Vậy tỉ số máy biến thế khi đó là:
\begin{equation*}
    x = \dfrac{n_{1}}{n_{2}} = \dfrac{U_{1}}{U_{2}} = \dfrac{6.10^3}{12.10^3} = \dfrac{1}{2}
\end{equation*}
\section*{\color[HTML]{4287f5}Câu 5: (2,0 điểm)}
Một điểm sáng S nằm ngoài trục chính và ở phía trước một thấu kính hội tụ có tiêu cự 20 cm, điểm sáng S cách trục chính 3 cm và cách thấu kính 40 cm như hình vẽ.

a) Hãy vẽ ảnh S’ của S tạo bởi thấu kính. Dùng kiến thức hình học để tính khoảng cách từ S’ đến thấu kính và S' đến trục chính.

b) Cho điểm sáng S dịch chuyển từ vị trí ban đầu theo phương song song trục chính với vận tốc không đổi 10 cm/s đến vị trí $S_{x}$ cách thấu kính 30 cm. Tính vận tốc của ảnh trong thời gian chuyển động.
\begin{figure}[H]
    \centering
    \includegraphics[width=0.5\linewidth]{tk.jpg}
    \label{fig:enter-label}
\end{figure}
\subsection*{\color[HTML]{4287f5}Lời giải:}
a)
\begin{figure}[H]
    \centering
    \includegraphics[width=0.7\linewidth]{img.jpg}
    \caption{Hình vẽ minh họa}
    \label{fig:enter-label}
\end{figure}
\noindent Xét $\triangle OHS \cong \triangle OH'S'$ ta có:\\
\begin{equation*}
    \implies \dfrac{OH}{OH'} = \dfrac{HS}{H'S'}
    \leftrightarrow \dfrac{d}{d'} = \dfrac{h}{h'} (1)
\end{equation*}
Xét $\triangle OIF \cong \triangle HSF$ ta có:\\
\begin{equation*}
    \implies \dfrac{HS}{OI} = \dfrac{HF}{OF}\\
    \leftrightarrow \dfrac{HS}{H'S'} = \dfrac{OH - OF}{OF}
    \leftrightarrow \dfrac{h}{h'} = \dfrac{d - f}{f} (2)
\end{equation*}\\
Từ (1) và (2) ta có:\\
\begin{equation*}
    \implies \dfrac{d}{d'} = \dfrac{d - f}{f}\\
    \leftrightarrow df = d'd - d'f\\
    \leftrightarrow \dfrac{1}{f} = \dfrac{1}{d} + \dfrac{1}{d'}
\end{equation*}\\
Mà ta có:
\begin{cases}
    f = 20 (cm)\\
    d = 40 (cm)
\end{cases}
$\implies d' = 40 (cm)$\\
\\
Từ (1) $\implies \dfrac{40}{40} = \dfrac{3}{h'} \implies h' = 3 (cm)$\\
\\
Vậy khoảng cách từ S' đến thấu kính là 40cm và S' đến trục chính là 3cm.\\
b)
Ta có:
\begin{equation*}
    \dfrac{1}{f} = \dfrac{1}{d} + \dfrac{1}{d'}\\
\end{equation*}
Mà $d_{1} = 30cm$\\
\begin{equation*}
    \implies \dfrac{1}{20} = \dfrac{1}{30} + \dfrac{1}{d'} \implies d' = 60 (cm)
\end{equation*}
Tương tự (1)  ta có:
\begin{equation*}
    \dfrac{d_{1}}{d_{1}'} = \dfrac{h}{h''} \leftrightarrow \dfrac{30}{60} = \dfrac{3}{h''}\\
    \implies h'' = 6 (cm)
\end{equation*}
Độ dịch chuyển của ảnh S' là:\\
\begin{equation*}
    S'S'' = \sqrt{(d_{1}' - d')^2 + (h'' - h')^2}\\
    \implies S'S'' = \sqrt{409} (cm)\\
\end{equation*}
Vận tốc của ảnh trong quá trình dịch chuyển là:\\
\begin{equation*}
    v' = \dfrac{S'S''}{t}
\end{equation*}
Mà:
\begin{equation*}
    t = \dfrac{d_{1} - d}{v} = \dfrac{40 - 30}{10} = 1 (s)
\end{equation*}
\begin{equation*}
    \implies v' = \dfrac{\sqrt{409}}{1} = \sqrt{409} (cm/s)
\end{equation*}
Vậy vận tốc của ảnh trong thời gian chuyển động là $\sqrt{409}$ (cm/s).
\section*{\color[HTML]{4287f5}Câu 6: (1,0 điểm)}
Một lọ thủy tinh có vỏ dày chứa đầy thủy ngân, được nút chặt bằng nút thủy tinh. Do thủy ngân là kim loại rất độc nên không thể đổ thủy ngân ra cân được. Bằng các dụng cụ sẵn có trong phòng thí nghiệm như sau:

- Một cái cân.

- Bình chia độ có giới hạn đo phù hợp bỏ lọt được lọ thủy tinh vào trong.

- Một lượng nước đủ dùng.

Biết khối lượng riêng của thủy tinh là $D_{1}$ và khối lượng riêng của thủy ngân là $D_{2}$ Hãy nêu một phương án để xác định khối lượng của thủy ngân trong lọ.
\subsection*{\color[HTML]{4287f5}Lời giải:}
Phương án xác định khối lượng của thuỷ ngân:\\
Cho lọ thủy ngân lên cân, ta cân được khối lượng tổng cộng:
\begin{equation*}
    m = m_{1} + m_{2} (1)
\end{equation*}
Với $m_{1}$, $m_{2}$ lần lượt là khối lượng lọ thủy tinh và khối lượng thủy ngân.\\
Thả lọ thủy tinh vào bình chia độ có lượng nước vừa đủ để bình chìm hoàn toàn. Thể tích nước dâng lên chính là thể tích cả lọ và thủy ngân bên trong:
\begin{equation*}
    V = V_{1} + V_{2} = \dfrac{m_{1}}{D_{1}} + \dfrac{m_{2}}{D_{2}} (2)
\end{equation*}
Với $V_{1}$, $V_{2}$ lần lượt là thể tích lọ thủy tinh và thể tích thủy ngân.\\
Từ (1) và (2), ta có:\\
\begin{equation*}
    m_{2} = \dfrac{(m - VD_{1})D_{2}}{D_{2} - D_{1}}
\end{equation*}

\end{document}
